\documentclass[
11pt, % The default document font size, options: 10pt, 11pt, 12pt
codirector, % Uncomment to add a codirector to the title page
]{simple_charter}

% El títulos de la memoria, se usa en la carátula y se puede usar el cualquier
% lugar del documento con el comando \ttitle
\titulo{Master test plan: Sistema de SLAM Visual/Inercial}

% Nombre del posgrado, se usa en la carátula y se puede usar el cualquier lugar
% del documento con el comando \degreename
\posgrado{Carrera de Especialización en Sistemas Embebidos}

% Tu nombre, se puede usar el cualquier lugar del documento con el comando
% \authorname
\autor{Ing. Gonzalo Gabriel Fernández}

%\documentname{Master test plan}

% El nombre del director y co-director, se puede usar el cualquier lugar del
% documento con el comando \supname y \cosupname y \pertesupname y \pertecosupname
\director{Dr. Ing. Fabio Ardiani}
\pertenenciaDirector{Nimble One}

%\codirector{John Doe} % para que aparezca en la portada se debe descomentar la
% opción codirector en el documentclass
%\pertenenciaCoDirector{FIUBA}

% Nombre del cliente, quien va a aprobar los resultados del proyecto, se puede
% usar con el comando \clientename y \empclientename
\cliente{Ing. Leandro Borgnino}
\empresaCliente{Fundación Fulgor}

% Nombre y pertenencia de los jurados, se pueden usar el cualquier lugar del
% documento con el comando \jurunoname, \jurdosname y \jurtresname y
% \perteunoname, \pertedosname y \pertetresname.
\juradoUno{Nombre y Apellido (1)}
\pertenenciaJurUno{pertenencia (1)}
\juradoDos{Nombre y Apellido (2)}
\pertenenciaJurDos{pertenencia (2)}
\juradoTres{Nombre y Apellido (3)}
\pertenenciaJurTres{pertenencia (3)}

\fechaINICIO{24 de agost de 2023}		%Fecha de inicio de la cursada de GdP \fechaInicioName
\fechaFINALPlan{14 de octubre de 2023} 	%Fecha de final de cursada de GdP
\fechaFINALTrabajo{15 de mayo de 2024}	%Fecha de defensa pública del trabajo final

\begin{document}

\maketitle
\thispagestyle{empty}
\pagebreak

\thispagestyle{empty}
{\setlength{\parskip}{0pt}
\setcounter{tocdepth}{2}
\tableofcontents{}
}
\pagebreak

\section*{Registros de cambios}
\label{sec:registros-de-cambios}

\begin{table}[ht]
\label{tab:registro}
\centering
\begin{tabularx}{\linewidth}{@{}|c|X|c|@{}}
\hline
\rowcolor[HTML]{C0C0C0}
Revisión &
\multicolumn{1}{c|}{\cellcolor[HTML]{C0C0C0}Detalles de los cambios realizados}
& Fecha
\\ \hline
0 & Creación del documento & 15 de octubre de 2023 \\ \hline
\end{tabularx}
\end{table}

\pagebreak

\section{1. Introducción}
\label{sec:1-introduccion}

En el presente documento se detalla la elección de estrategias de testing para el software
desarrollado para el sistema de odometría visual-inercial basado en SLAM propuesto en la planificación presentada en el documento [PYPH-DOC-001-R4], con el objetivo de
concientizar a todos los actores vinculados con el proyecto sobre los riesgos que deben cubrirse y
acordar cuántas pruebas deben realizarse y cuándo dentro del proceso de desarrollo de software.

Como se describe en el documento de arquitectura de software del sistema [PYPH-DOC-003-R2], el
software del sistema puede dividirse en los siguientes módulos:

\begin{itemize}
	\item Driver de la Unidad de Medición Inercial (IMU por sus siglas en  inglés), con el que se
	obtienen las lecturas del acelerómetro, giróscopo y magnetómetro y se transfieren a la aplicación.
	\item Driver de la cámara fotográfica, con el que se obtienen las imágenes de la cámara
	fotográfica y se transfieren a la aplicación.
	\item Algoritmo de SLAM, con el que se procesan los datos obtenidos con la IMU y la
	cámara fotográfica y se obtiene como salida un mapa del entorno, la posición y la orientación del
	sistema físco.
	\item Capa de comunicación con micro-ROS, con el que se transmiten los datos adquiridos con el
	sistema SLAM a un sistema externo de control, mediante la publicación en el canal de comunicación
	correspondiente del sistema ROS 2.
\end{itemize}

En la figura \ref{fig:software-arch} se observa un diagrama de bloques presentado en el documento de
arquitectura de software mencionado previamente, obtenido mediante la segmentación del proceso.

\begin{figure}[ht]
	\centering
	\includegraphics[width=\textwidth]{imgs/polyphemus_pipeline.png}
	\caption{Arquitectura de software de sistema SLAM}
	\label{fig:software-arch}
\end{figure}

En las siguientes secciones del documento, se selecciona las carcterísticas de calidad e evaluar
mediante  el testing del software desarrollado, se determina la importancia relativa de dichas
características y se define que características serán evaluadas en cada nivel de testing a realizar.
Además, se  divide el software en subsistemas, siguiendo el diagrama de la figura
\ref{fig:software-arch}, se determina la importancia relativa de cada uno y cuál será la técnica de
test a emplear en cada uno.

\section{2. Asignaciones}
\label{sec:2-asignaciones}

\subsection{Responsable}
\label{ssec:responsable}

El \textbf{responsable} de la elaboración del documento es el Ing. Gonzalo Gabriel Fernandez,
también a cargo del desarrollo del proyecto.

\subsection{Contratista}
\label{ssec:contratista}

El Ing. Gonzalo Gabriel Fernandez también toma el rol de \textbf{contratista} a cargo del testing
del software elaborado para el sistema SLAM.

\subsection{Alcance}
\label{ssec:alcance}

El alcance del test de aceptación es:

\begin{itemize}
	\item El driver de la Unidad de Medición Inercial
	\item El driver de la cámara fotográfica
	\item El algoritmo de SLAM
	\item La capa de comunicación con micro-ROS
\end{itemize}

\subsection{Objetivos}
\label{ssec:objetivos}

Los objetivos son:

\begin{itemize}
	\item Determinar si el sistema cumple los requerimientos
	\item Reportar y documentar las diferencias entre el comportamiento observado y el
	comportamiento esperado.
	\item Obtener una plataforma de testing que pueda ser reutilizada en futuras releases del
	proyecto.
\end{itemize}

\subsection{Precondiciones (externas)}
\label{ssec:precondiciones-externas}

Las precondiciones son:

\begin{itemize}
	\item El documento de requerimientos de software debe estar completo el mes posterior a la fecha
	de inicio del proyecto.
	\item La plataforma de testing estático, chequeo autómatico de código, generación automática de
	documentación debe ser funcional el mes posterior a la fecha de entraga del documento de
	requerimientos de software.
	\item Los tests deben estar finalizados el 26 de Enero de 2024.
\end{itemize}

\subsection{Precondiciones (internas)}
\label{ssec:precondiciones-internas}

Se necesitan las siguientes precondiciones para que las tareas de testing sean concluidas en tiempo
y forma:

\begin{itemize}
	\item El desarrollo y posterior ejecución de un test se lleva a cabo una vez finalizada la
	documentación del software involucrado.
	\item Todas las demoras en el desarrollo de tests deben ser documentadas y comunicadas al equipo
	de desarrollo.
	\item  Todos los cambios sobre un test serán documentados.
	\item Todas las herramientas e infraestructura necesaria para testing estará disponible durante
	el tiempo asignado a ejecucón de los tests.
\end{itemize}

\section{3. Bases del testing}
\label{sec:3-bases-del-testing}

Los tests detallados tienen como base los siguientes documentos:

\begin{itemize}
	\item \textbf{[PYPH-DOC-001-R4]:} Planificación de proyecto: Sistema de SLAM Visual/Inercial
	\item \textbf{[PYPH-DOC-002-R2]:} Especificación de requerimientos de software: Sistema de SLAM
	Visual/Inercial
	\item \textbf{[PYPH-DOC-003-R2]:} Arquitectura de software: Sistema de SLAM Visual/Inercial
\end{itemize}

\section{4. Estrategia de testing}
\label{sec:4-estrategia-de-testing}

\subsection{Selección de características de calidad}
\label{ssec:seleccion-de-caracteristicas-de-calidad}

La tabla \ref{tab:qualty-char} expone las características de calidad que deben ser evaluadas
y su importancia relativa (IR).

\begin{table}[ht]
\centering
\begin{tabular}{@{}lll@{}}
\toprule
\textbf{Características de calidad} & \textbf{Subcaracterística} & \textbf{IR (\%)} \\ \midrule
Funcionalidad & Idoneidad e interoperabilidad & 30 \\
Usabilidad & Comprensibilidad y operabilidad & 20 \\
Mantenibilidad & Posibilidad de cambiar y testeabilidad & 30 \\
Portabilidad & Adaptabilidad & 20 \\
Total & & 100 \\ \bottomrule
\end{tabular}
\caption{Importancia relativa de las características de calidad}
\label{tab:qualty-char}
\end{table}

A continuación se detallan las subcaracterísticas de calidad seleccionadas:

\begin{itemize}
	\item Idoneidad (\textit{Suitability}): las funcionalidades que provee el software deben ser las
	necesarias para que el sistema cumpla con todos los requisitos de funcionamiento establecidos en
	el documento [PYPH-DOC-002-R2].
	\item Interoperabilidad (\textit{Inteoperability}): los distintos módulos del software deben ser
	independientes y proveer interfaces claras y documentadas, de tal forma de permitir su
	reutilización en el proyecto actual, proyectos futuros o proyectos de terceros.
	\item Comprensibilidad (\textit{Understandability}): los distintos módulos del software y sus
	interfaces deben estar documentados de forma clara con ejemplos de uso que permitan al usuario
	adaptarlos a su sistema.
	\item Operabilidad (\textit{Operability}): un usuario con conocimientos en ROS 2 debe poder
	interactuar con el sistema rápidamente una vez interpretada la documentación del software del
	proyecto.
	\item Posibilidad de cambiar (\textit{Changeability}): tratándose de un proyecto que tendrá como
	usuario investigadores, es importante que el proyecto pueda ser modificado y extendido. Por lo
	tanto, el software debe estar completamente documentado y debe poseer ejemplos de extensión o uso
	de módulos de forma independiente.
	\item Testeabilidad (\textit{Testability}): el software debe ser desarrollo con un constante foco
	en su capacidad para evaluarlo.
	\item Adaptabilidad (\textit{Adaptability}):  se desea que el software sea portable a otras
	plataformas embebidas, y por lo tanto deben existir interfaces claras para portabilidad.
\end{itemize}

\subsection{Asignación de las características de calidad a cada nivel de prueba}
\label{ssec:asignacion-de-las-caracteristicas-de-calidad-a-cada-nivel-de-prueba}

A continuación se indica las características de calidad y su importancia relativa para cada nivel de
testing.

\subsubsection{Test unitario}
\label{sssec:test-unitario}

\begin{table}[H]
\centering
\begin{tabular}{@{}lc@{}}
\toprule
\textbf{Características de calidad} & \textbf{Importancia relativa (\%)} \\ \midrule
Posibilidad de cambiar              & 40                                 \\
Testeabilidad                       & 60                                 \\
Total                               & 100                                \\ \bottomrule
\end{tabular}
\caption{Importancia de las características de calidad en el test unitario}
\label{tab:ir-unit}
\end{table}

\subsubsection{Test de integración de software}
\label{sssec:test-de-integracion-de-software}

\begin{table}[H]
\centering
\begin{tabular}{@{}lc@{}}
\toprule
\textbf{Características de calidad} & \textbf{Importancia relativa (\%)} \\ \midrule
Interoperabilidad                   & 10                                 \\
Comprensibilidad                    & 20                                 \\
Posibilidad de cambiar              & 10                                 \\
Testeabilidad                       & 30                                 \\
Adaptabilidad                       & 30                                 \\
Total                               & 100                                \\ \bottomrule
\end{tabular}
\caption{Importancia de las características de calidad en el test de integración de software}
\label{tab:ir-int-soft}
\end{table}

\subsubsection{Test de integración de software con hardware}
\label{sssec:test-de-integracion-de-software-con-hardware}

\begin{table}[H]
\centering
\begin{tabular}{@{}lc@{}}
\toprule
\textbf{Características de calidad} & \textbf{Importancia relativa (\%)} \\ \midrule
Idoneidad                           & 20                                 \\
Comprensibilidad                    & 30                                 \\
Posibilidad de cambiar              & 5                                  \\
Testeabilidad                       & 5                                  \\
Adaptabilidad                       & 40                                 \\
Total                               & 100                                \\ \bottomrule
\end{tabular}
\caption{Importancia de las características de calidad en el test de integración de software con
hardware}
\label{tab:ir-int-soft-hard}
\end{table}

\subsubsection{Test de sistema}
\label{sssec:test-de-sistema}

\begin{table}[H]
\centering
\begin{tabular}{@{}lc@{}}
\toprule
\textbf{Características de calidad} & \textbf{Importancia relativa (\%)} \\ \midrule
Idoneidad                           & 50                                 \\
Comprensibilidad                    & 20                                 \\
Operabilidad                        & 30                                 \\
Total                               & 100                                \\ \bottomrule
\end{tabular}
\caption{Importancia de las características de calidad en el test de sistema}
\label{tab:ir-test-sys}
\end{table}

\subsubsection{Test de aceptación}
\label{sssec:test-de-aceptacion}

\begin{table}[H]
\centering
\begin{tabular}{@{}lc@{}}
\toprule
\textbf{Características de calidad} & \textbf{Importancia relativa (\%)} \\ \midrule
Idoneidad                           & 60                                 \\
Operabilidad                        & 40                                 \\
Total                               & 100                                \\ \bottomrule
\end{tabular}
\caption{Importancia de las características de calidad en el test de aceptación}
\label{tab:ir-test-accept}
\end{table}

\subsubsection{Test de campo}
\label{sssec:test-de-campo}

\begin{table}[H]
\centering
\begin{tabular}{@{}lc@{}}
\toprule
\textbf{Características de calidad} & \textbf{Importancia relativa (\%)} \\ \midrule
Idoneidad                           & 60                                 \\
Operabilidad                        & 40                                 \\
Total                               & 100                                \\ \bottomrule
\end{tabular}
\caption{Importancia de las características de calidad en el test de campo}
\label{tab:ir-test-field}
\end{table}

\subsection{Asignación de niveles de prueba a las características de calidad}
\label{ssec:asignacion-de-niveles-de-prueba-a-las-caracteristicas-de-calidad}

En la tabla \ref{tab:test-level-quality} se observan los niveles de prueba asignados a las distintas
características de calidad detalladas en la sección
\ref{ssec:asignacion-de-las-caracteristicas-de-calidad-a-cada-nivel-de-prueba}.

\begin{table}[H]
\centering
\begin{tabular}{@{}llllllll@{}}
\toprule
\textbf{} & \textbf{(1)} & \textbf{(2)} & \textbf{(3)} & \textbf{(4)} & \textbf{(5)} & \textbf{(6)} & \textbf{(7)} \\ \midrule
\textbf{Importancia relativa (\%)} & 15 & 15 & 10 & 10 & 10 & 20 & 20 \\
\textbf{Test unitario} &  &  &  &  & ++ & ++ &  \\
\textbf{\begin{tabular}[c]{@{}l@{}}Test de integración de software\end{tabular}}
 &  & ++ & ++ &  & ++ & ++ & ++ \\
\textbf{Test de integración de software y hardware} & + &  & ++ &  & + & + & ++ \\
\textbf{Test de sistema} & + &  & + & ++ &  &  &  \\
\textbf{Test de aceptación} & ++ &  &  & ++ &  &  &  \\
\textbf{Test de campo} & ++ &  &  & ++ &  &  &  \\ \bottomrule
\end{tabular}
\caption{Asignación de nivel de test por característica de calidad}
\label{tab:test-level-quality}
\end{table}

Donde los números corresponden a las siguientes características de calidad:
\begin{enumerate}
	\item Idoneidad
	\item Interoperabilidad
	\item Comprensibilidad
	\item Operabilidad
	\item Posibilidad de cambiar
	\item Testeabilidad
	\item Adaptabilidad
\end{enumerate}

Donde las simbología representa lo siguiente:
\begin{itemize}
	\item ++ : la característica de calidad será cubierta completamente ya que es un objetivo
	importante en ese nivel de testing.
	\item + : el nivel de testing cubrirá la característica de calidad.
	\item vacío: la característica de calidad no es un problema en ese nivel de testing
\end{itemize}

Los motivos de la asignación de niveles de prueba a las características de calidad seleccionadas son
los siguientes:

\begin{itemize}
	\item Idoneidad: es importante en la integración de software con hardware, en las pruebas de
	sistema, en las pruebas de aceptación y en las pruebas de campo, donde se comprueba que
	efectivamente el software desarrollado cumple los requerimientos planteados en el documento de
	ERS [PYPH-DOC-002-R2].
	\item Interoperabilidad: en el test de integración de software es donde se comprobará la
	independencia de los distintos módulos que componen el sistema, la arquitectura de sus
	interfaces y su documentación.
	\item Comprensibilidad: será evaluada principalmente en la integración de software, en la
	integración de software con hardware y en las pruebas de sistema, poniendo foco en la experiencia
	del usuario al utilizar el software.
	\item Operabilidad: en las pruebas de sistema se analiza la interacción con el sistema externo
	con ROS 2, y se evaluará la facilidad de uso del sistema.
	\item Posibilidad de cambiar: será foco en las pruebas unitarias y en la integración de software
	analizando la claridad y escalabilidad del software desarrollado.
	\item Testeabilidad: especialmente importante a bajo nivel, en las pruebas unitarias y de
	integración de software. Es imprescindible que el software desarrollado sea fácil de evaluar y
	compatible con una plataforma de testing automático.
	\item Adaptabilidad: en las pruebas de integración de software e integración de software con
	hardware es donde se evaluará la facilidad de portación a otras arquitecturas de procesador y
	otros sistemas embebidos.
\end{itemize}

\section{5. Estrategias por nivel de prueba}
\label{sec:5-estrategias-por-nivel-de-prueba}

A continuación, se divide elsistema en subsistemas, se les asigna una importancia relativa y se
define la importancia de las característica de calidad en cada uno de ellos.
Para finalizar, se define qué técnicas de test serán utilizadas en cada subsistema.

\subsection{División del sistema en subsistemas}
\label{ssec:division-del-sistema-en-subsistemas}

El software del sistema SLAM puede dividirse en los siguientes subsistemas:

\begin{itemize}
	\item \textbf{Subsistema A: driver de la IMU}, con el que se obtienen las lecturas del acelerómetro, giróscopo y magnetómetro y se transfieren a la aplicación.
	\item \textbf{Subsistema B: driver de la cámara fotográfica}, con el que se obtienen las imágenes de la cámara fotográfica y se transfieren al módulo de procesamiento de imágenes.
	\item \textbf{Subsistema C: algoritmo de SLAM}, con el que se procesan los datos obtenidos de la
	IMU y la cámara fotográfica y se obtiene como salida un mapa del entorno, la posición y la
	orientación del sistema físico.
	\item \textbf{Subsistema D: capa de comunicación con micro-ROS}, con el que se transmiten los
	datos adquiridos con el sistema SLAM a un sistema externo de control, mediante la publicación
	en el canal de comunicación correspondiente del sistema ROS 2.
\end{itemize}

\subsection{Importancia relativa de los subsistemas}
\label{ssec:importancia-relativa-de-los-subsistemas}

En la tabla \ref{tab:ir-subsys} se observa la importancia relativa asignada a cada subsistema
del proyecto.

\begin{table}[ht]
\centering
\begin{tabular}{@{}lc@{}}
\toprule
Subsistema & Importancia relativa (\%) \\ \midrule
Driver de la IMU & 30 \\
Driver de la cámara fotográfica & 30 \\
Algoritmo de SLAM & 30 \\
Capa de comunicación con micro-ROS & 10 \\
Total & 100 \\ \bottomrule
\end{tabular}
\caption{Importancia relatica de los subsistemas}
\label{tab:ir-subsys}
\end{table}

\subsection{Importancia de test por combinación de subsistema y característica de calidad}
\label{ssec:importancia-de-test-por-combinacion-de-subsistema-y-caracteristica-de-calidad}

En la tabla \ref{tab:quality-subsys} se observa la importancia de cada característica de
calidad seleccionada en cada subsistema en específico.

\begin{table}[ht]
\centering
\begin{tabular}{@{}lllll@{}}
\toprule
 & Sub. A & Sub. B & Sub. C & Sub. D \\ \midrule
\rowcolor[HTML]{EFEFEF}
Importancia relativa (\%) & 30 & 30 & 30 & 10 \\
Idoneidad & ++ & ++ & ++ & + \\
Interoperabilidad & ++ & ++ & + &  \\
Comprensibilidad & ++ & ++ & + &  \\
Operabilidad &  &  &  & ++ \\
Posibilidad de cambiar & + & + & ++ & + \\
Testeabilidad & ++ & ++ & ++ &  \\
Adaptabilidad & ++ & ++ & + &  \\ \bottomrule
\end{tabular}
\caption{Importancia de características de calidad en cada subsistema}
\label{tab:quality-subsys}
\end{table}

Donde las simbología representa lo siguiente:
\begin{itemize}
	\item ++ : La característica de calidad será cubierta completamente ya que es un objetivo
	importante en ese subsistema.
	\item + : La característica de calidad se cubrirá en el subsistema.
	\item vacío: La característica de calidad no es necesaria en el subsistema.
\end{itemize}

\subsection{Técnicas de test a ser utilizadas}
\label{ssec:tecnicas-de-test-a-ser-utilizadas}

Las técnicas de test a utilizar son:

\begin{itemize}
	\item \textit{Elementary Comparison Test} (ECT)
	\item \textit{Classification-Tree Method} (CTM)
	\item \textit{Control Flow Test} (CFT)
	\item \textit{State Transition Testing} (STT)
\end{itemize}

Se aplicarán la técnicas de test mencionadas a cada uno de los subsistemas, de acuerdo a las tabla
\ref{tab:tec-subsys}.

\begin{table}[H]
\centering
\begin{tabular}{@{}lcccc@{}}
\toprule
Técnica de test & Sub. A & Sub. B & Sub. C & Sub. D \\ \midrule
ECT & + & + &  &  \\
CTM &  &  & + &  \\
CFT & + & + & + & + \\
STT &  &  &  & + \\ \bottomrule
\end{tabular}
\caption{Técnicas de test a aplicar en cada subsistema}
\label{tab:tec-subsys}
\end{table}

\end{document}
